\documentclass{standalone}
\usepackage{tikz}
\usetikzlibrary{positioning}
\usepackage[T1]{fontenc} % necessary for italic font in title
\usepackage{hyperref} % for referencing
\usepackage{xcolor} % own color definitions
\usepackage{listings} % code snippets
\usepackage[normalem]{ulem} % strikethrough text
\usepackage[mathscr]{eucal} % tensor euler script
\usepackage{amsmath}
\usepackage{tabularx}
\usepackage{commath}
\usepackage{graphicx}
\usepackage[caption = false]{subfig}
\usepackage{amssymb}
\usetikzlibrary{patterns}
\usetikzlibrary{decorations.pathreplacing}
\usetikzlibrary{arrows}
\usepackage{tcolorbox}
\usetikzlibrary{calc}
\definecolor{mpiblue}{HTML}{33a5c3}
\colorlet{MPIblue}{mpiblue}
\definecolor{mpibluefont}{HTML}{17a1c1}
\colorlet{MPIbluefont}{mpibluefont}
\definecolor{mpigreen}{HTML}{007675}
\colorlet{MPIgreen}{mpigreen}
\definecolor{mpired}{HTML}{78004B}
\colorlet{MPIred}{mpired}
\definecolor{mpisand}{HTML}{ece9d4}
\colorlet{MPIsand}{mpisand}
\newcommand*\circledb[1]{\tikz[baseline=(char.base)]{
		\node[shape=circle,draw,inner sep=2pt,fill = mpiblue] (char) {#1};}}
\newcommand*\circled[1]{\tikz[baseline=(char.base)]{
		\node[shape=circle,draw,inner sep=2pt,fill = mpired!50] (char) {#1};}}    
\begin{document}
	\scalebox{0.7}{
		\begin{tikzpicture}
			%%%%%%%%%%%%%%%%%%%%%%%%%%%%%%%%%%%%%%%%%%%%%%%%%%%%%%%%%%%%%%%%%%%%%%%%%%%%%%%%%%%%%%%%%%%%
			%%Tucker-core%%%
			\coordinate (A1) at (0em,0cm,0cm); % central top point (To pick)
			\coordinate (A2) at (1.2cm,0cm,0cm);
			\coordinate (A3) at (0em,0.8cm,0cm); % central bottom point (To pick)
			\coordinate (A4) at (0em,0cm,0.9cm);
			\coordinate (A5) at (1.2cm,0.8cm,0cm);
			\coordinate (A6) at (1.2cm,0cm,0.9cm);
			\coordinate (A7) at (0em,0.8cm,0.9cm);
			\coordinate (A8) at (1.2cm,0.8cm,0.9cm);
			
			%% Possibly draw front lines
			\draw[thick] (A3) -- (A5);
			\draw[thick] (A3) -- (A7);
			\draw [line width = 15pt,decorate,decoration={brace,amplitude=70pt},xshift=300pt,yshift=5pt]
			(A2) -- (A6) node[above right = 4cm and 10cm]
			{\footnotesize {\scalebox{18}{$R_3$}}};
			\draw [line width = 15pt,decorate,decoration={brace,amplitude=70pt,mirror,raise=4pt},xshift=300pt,yshift=5pt]
			(A4) -- (A6) node[below left = 4cm and 18cm]{\footnotesize {\scalebox{18}{$R_2$}}}; %node[below = 1cm, xshift = -30cm ] {\scalebox{15.5}{$I_2$}};
			\draw[thick] (A8) -- (A5);
			\draw[thick] (A8) -- (A7);
			\draw[thick] (A8) -- (A6);
			\draw [line width = 15pt,decorate,decoration={brace,amplitude=70pt},xshift=300pt,yshift=5pt]
			(A4) -- (A7) node[below left = 18cm and 3cm]
			{\footnotesize {\scalebox{18}{$R_1$}}}; %node[below = 13cm,left = 0.4cm ] {\scalebox{15.5}{$r_1$}};
			\draw[thick,dashed] (A1) -- (A2) ;
			\draw[thick,dashed] (A1) -- (A4); % node[above = 7cm, right = 6cm ] {\scalebox{15.5}{$r_2$}};
			\draw[thick,dashed] (A1) -- (A3);
			
			
			%% Possibly draw back faces
			\shadedraw[inner color=white,outer color=mpigreen,draw=black] (A1) -- (A2) -- (A5) -- (A3) -- cycle; % face6	
			\shadedraw[inner color=white,outer color=mpigreen,draw=black,opacity = 0.6] (A1) -- (A4) -- (A7) -- (A3) -- cycle; % face 3	
			\shadedraw[inner color=white,outer color=mpigreen,draw=black, opacity = 0.6] (A2) -- (A5) -- (A8) -- (A6) -- cycle; % face 3
			\shadedraw[inner color=white,outer color=mpigreen,draw=black, opacity = 0.5] (A4) -- (A6) -- (A8) -- (A7) -- cycle; % face 4
			\shadedraw[inner color=white,outer color=mpigreen,draw=black,opacity=0.2] (A1) -- (A2) -- (A6) -- (A4) -- cycle; % f2
			\shadedraw[inner color=white,outer color=mpigreen,draw=black,opacity=0.2] (A3) -- (A5) -- (A8) -- (A7) -- cycle; % f5
			% center
			\node (G) at (barycentric cs:A4=1,A6=1,A8=1,A7=1) {};
			\node (G1) [right =  2 cm] at (G) {\scalebox{25}{$\boldsymbol{\mathscr{G}}$}};
			% %%%%%%%%%%%%%%%%%%%%%%%%%%%%%%%%%%%%%%%%%%%%%%%%%%%%%%%%%%%%%%%%%%%%%%%%%%%%%%%%%%%%%%%%%%%%%
			%% 1st Tucker Matrix Core %%
			% B7 ------------ B8
			% |                        |
			% |        C1           |
			% |                        |
			%B5 -------------- B6
			
			\coordinate (B5) at (-1cm,0cm,2cm);
			\coordinate (B6) at (-0.2cm,0cm,2cm);
			\coordinate (B7) at (-1cm,1.2cm,2cm);
			\coordinate (B8) at (-0.2cm,1.2cm,2cm);
			
			%% Possibly draw front lines
			\draw [line width = 15pt,decorate,decoration={brace,amplitude=70pt,mirror,raise=4pt},xshift=300pt,yshift=5pt]
			(B5) -- (B6) node[below left = 4cm and 8cm]{\footnotesize {\scalebox{18}{$R_1$}}}; % below horizontal 
			\draw [line width = 15pt,decorate,decoration={brace,amplitude=70pt},xshift=300pt,yshift=5pt]
			(B5) -- (B7) node[below left = 15cm and 3cm]
			{\footnotesize {\scalebox{18}{$I_1$}}};
			\draw[thick] (B8) -- (B7); % above horizontal
			\draw[thick] (B8) -- (B6); % right verticle
			
			\shadedraw[inner color=white,outer color=mpired!70,draw=black] (B6) -- (B8) -- (B7) -- (B5) -- cycle; % coloring A^(1)
			% center
			\node (C1) at (barycentric cs:B6=1,B8=1,B7=1,B5=1) {};
			\node (C1') [] at (C1) {\scalebox{15}{$\boldsymbol{A}^{(1)}$}};
			% %%%%%%%%%%%%%%%%%%%%%%%%%%%%%%%%%%%%%%%%%%%%%%%%%%%%%%%%%%%%%%%%%%%%%%%%%%%%%%%%%%%%%%%%%%%%%%
			%% 1st Tucker Matrix Core %%
			% C7 ----------------------------C8
			% |                   D1                     |
			% |                                             |
			%C5 ----------------------------- C6
			
			\coordinate (C5) at (1.8cm,0cm,0cm);
			\coordinate (C6) at (3cm,0cm,0cm);
			\coordinate (C7) at (1.8cm,0.7cm,0cm);
			\coordinate (C8) at (3cm,0.7cm,0cm);
			
			%% Possibly draw front lines
			\draw [line width = 15pt,decorate,decoration={brace,amplitude=70pt,mirror,raise=4pt},xshift=300pt,yshift=5pt]
			(C5) -- (C6) node[below left = 4cm and 14cm]{\footnotesize {\scalebox{18}{$I_2$}}}; % below horizontal 
			\draw [line width = 15pt,decorate,decoration={brace,amplitude=70pt},xshift=300pt,yshift=5pt]
			(C8) -- (C6) node[above right = 7cm and 3cm]
			{\footnotesize {\scalebox{18}{$R_2$}}};
			\draw[thick] (C8) -- (C7); % above horizontal
			\draw[thick] (C5) -- (C7); % right verticle
			
			\shadedraw[inner color=white,outer color=orange!60,draw=black] (C6) -- (C8) -- (C7) -- (C5) -- cycle; % coloring A^(1)
			% center
			\node (D1) at (barycentric cs:C6=1,C8=1,C7=1,C5=1) {};
			\node (D1') [] at (D1) {\scalebox{15}{$\boldsymbol{A}^{(2)}$}};
			% %%%%%%%%%%%%%%%%%%%%%%%%%%%%%%%%%%%%%%%%%%%%%%%%%%%%%%%%%%%%%%%%%%%%%%%%%%%%%%%%%%%%%%%%%%%%%%
			%%Main tensor X %%
			
			\coordinate (x1) at (-4.1cm,-0.3cm,0cm); % central top point (To pick)
			\coordinate (x2) at (-2.7cm,-0.3cm,0cm);
			\coordinate (x3) at (-4.1cm,1.1cm,0cm); % central bottom point (To pick)
			\coordinate (x4) at (-4.1cm,-0.3cm,1.4cm);
			\coordinate (x5) at (-2.7cm,1.1cm,0cm);
			\coordinate (x6) at (-2.7cm,-0.3cm,1.4cm);
			\coordinate (x7) at (-4.1cm,1.1cm,1.4cm);
			\coordinate (x8) at (-2.7cm,1.1cm,1.4cm);
			
			%% Possibly draw front lines
			\draw[thick] (x3) -- (x5);
			\draw[thick] (x3) -- (x7);
			\draw[thick] (x2) -- (x6); 
			\draw[thick] (x4) -- (x6);
			\draw[thick] (x8) -- (x5);
			\draw[thick] (x8) -- (x7);
			\draw[thick] (x8) -- (x6);
			\draw[thick] (x4) -- (x7); 
			\draw[thick,dashed] (x1) -- (x2) ;
			\draw[thick,dashed] (x1) -- (x4); % 
			\draw[thick,dashed] (x1) -- (x3);
			%% Possibly draw back faces
			
			\shadedraw[inner color=white,outer color=mpiblue,draw=black] (x1) -- (x2) -- (x5) -- (x3) -- cycle; % face 6
			
			\shadedraw[inner color=white,outer color=mpiblue,draw=black,opacity = 0.6] (x1) -- (x4) -- (x7) -- (x3) -- cycle; % face 3
			
			\shadedraw[inner color=white,outer color=mpiblue,draw=black, opacity = 0.6] (x2) -- (x5) -- (x8) -- (x6) -- cycle; % face 3
			
			\shadedraw[inner color=white,outer color=mpiblue,draw=black, opacity = 0.5] (x4) -- (x6) -- (x8) -- (x7) -- cycle; % face 4
			
			\shadedraw[inner color=white,outer color=mpiblue,draw=black,opacity=0.5] (x1) -- (x2) -- (x6) -- (x4) -- cycle; % f2
			
			\shadedraw[inner color=white,outer color=mpiblue,draw=black,opacity=0.5] (x3) -- (x5) -- (x8) -- (x7) -- cycle; % f5
			\node (X1) at (barycentric cs:x3=1,x5=1,x8=1,x7=1) {};
			\node (X1') [below = 17cm] at (X1) {\scalebox{25}{$\boldsymbol{\mathscr{X}}$}};
			
			
			\draw [line width = 15pt,decorate,decoration={brace,amplitude=70pt,mirror,raise=4pt},xshift=300pt,yshift=5pt]
			(x4) -- (x6) node[below left = 4cm and 18cm]
			{\footnotesize {\scalebox{18}{$I_2$}}}; 
			\draw [line width = 15pt,decorate,decoration={brace,amplitude=70pt},xshift=300pt,yshift=5pt]
			(x4) -- (x7) node[below left = 18cm and 3cm]
			{\footnotesize {\scalebox{18}{$I_1$}}};
			\draw [line width = 15pt,decorate,decoration={brace,amplitude=70pt},xshift=300pt,yshift=5pt]
			(x2) -- (x6) node[above right = 4cm and 10cm]
			{\footnotesize {\scalebox{18}{$I_3$}}};
			
			\node [inner sep=0pt,below right = 12cm and 32cm] at (X1) {\scalebox{30}{$\cong$}};
			
			%% 3rd Tucker Matrix Core %%
			%           D5 ------------ D6
			%           /                    /
			%        /         E1      /
			%       /                    /
			% D7 ------------ D8
			
			
			\coordinate (D5) at (1cm,2.4cm,1.5cm);
			\coordinate (D6) at (2cm,2.4cm,1.5cm);
			\coordinate (D7) at (0.7cm,2cm,2cm);
			\coordinate (D8) at (1.7cm,2cm,2cm);
			
			%% Possibly draw front lines
			\draw [line width = 15pt,decorate,decoration={brace,amplitude=70pt,mirror,raise=4pt},xshift=300pt,yshift=5pt]
			(D7) -- (D8) node[below left = 4cm and 8cm]{\footnotesize {\scalebox{18}{$R_3$}}}; % below horizontal 
			\draw[thick]  (D5) -- (D7);
			
			\draw[thick] (D5) -- (D6); % above horizontal
			\draw [line width = 15pt,decorate,decoration={brace,amplitude=70pt},xshift=300pt,yshift=5pt] (D6) -- (D8) node[above right = 2cm and 10cm] {\footnotesize {\scalebox{18}{$I_1$}}}; % right verticle
			
			\shadedraw[inner color=white,outer color=brown!70,draw=black] (D6) -- (D8) -- (D7) -- (D5) -- cycle; % coloring A^(1)
			% center
			\node (E1) at (barycentric cs:D6=1,D8=1,D7=1,D5=1) {};
			\node (E1') [] at (E1) {\scalebox{15}{$\boldsymbol{A}^{(3)}$}};
			
		\end{tikzpicture}
	}
	\end{document}